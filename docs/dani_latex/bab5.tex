%!TEX root = ./template-skripsi.tex
%-------------------------------------------------------------------------------
%                            	BAB IV
%               		KESIMPULAN DAN SARAN
%-------------------------------------------------------------------------------

\chapter{KESIMPULAN DAN SARAN}

\section{Kesimpulan}
Berdasarkan hasil implementasi aplikasi yang telah dirancang, maka diperoleh kesimpulan sebagai berikut:

\begin{enumerate}
	\item Aplikasi pengkajian luka kronis versi kedua berbasis Android berhasil dikembangkan dengan mengintegrasikan berbagai fitur dan hasil penelitian sebelumnya ke dalam Product Backlog. Proses perancangannya mengikuti metode Scrum, yang mencakup tahapan penyusunan Product Backlog, Sprint Backlog, serta pelaksanaan pengembangan dalam sembilan Sprint.
	
	\item Dengan menerapkan Fragment, aplikasi dapat memiliki tampilan yang lebih dinamis dan fleksibel tanpa harus membuat banyak Activity, sehingga mengurangi overhead dan meningkatkan performa. Selain itu, Fragment memungkinkan penggunaan kembali komponen UI di berbagai bagian aplikasi, mendukung navigasi yang lebih lancar, serta memudahkan implementasi desain responsif, terutama dalam aplikasi berbasis Android modern.
\end{enumerate}

\section{Saran}
Adapun saran untuk penelitian selanjutnya adalah:
\begin{enumerate} 
	\item Berdasarkan diskusi dengan product owner, harus dimulainya pengembangan sistem berikutnya yang memiliki fitur tagihan agar dapat membantu mempercepat antrean pasien.
	\item Berdasarkan diskusi dengan scrum master, perlu ditambahkan feedback dari pelayanan yang dijalankan oleh klinik setelah pasien selesai berobat, apakah pelayanan yang dilakukan baik atau buruk.
\end{enumerate}


% Baris ini digunakan untuk membantu dalam melakukan sitasi
% Karena diapit dengan comment, maka baris ini akan diabaikan
% oleh compiler LaTeX.
\begin{comment}
\bibliography{daftar-pustaka}
\end{comment}