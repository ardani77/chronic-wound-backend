\chapter*{\centering{\large{ABSTRAK}}}

\begin{spacing}{1}
\textbf{MUHAMMAD ARDANI}. Aplikasi Pengkajian Luka Berbasis Android Terintegrasi Dengan Sistem Informasi Klinik Keperawatan Luka. Skripsi. Program Studi Ilmu Komputer Fakultas Matematika dan Ilmu Pengetahuan Alam, Universitas Negeri Jakarta. 2025. Di bawah bimbingan Muhammad Eka Suryana, M.Kom dan Drs. Mulyono, M.Kom.
\newline
\newline
Penyebab luka kronis dan faktor yang mempengaruhi penyembuhan luka cukup kompleks dan beragam. Dalam proses diagnosis, pengobatan dan manajemen luka kronis, dokter dan perawat perlu memiliki pemahaman menyeluruh serta kelengkapan rekam medis, manajemen yang efektif dan pemantauan kondisi fisik umum pasien secara tepat waktu, laporan pemeriksaan laboratorium, penilaian dan pengobatan luka fraksional. Secara umum, proses pengkajian luka masih dilakukan secara tradisional dengan metode pencatatan atau pengarsipan berbasis kertas. Saat ini, data-data tersebut masih dicatat secara manual, yang berisiko menimbulkan kesalahan dalam perhitungan. Selain itu, penelitian sebelumnya memiliki kekurangan penilaian data pengkajian luka kronis secara akurat. Sehingga diperlukannya penelitian ini dengan tujuan untuk membuat aplikasi pengkajian luka kronis dengan modul \textit{Bates-Jensen Assessment Wound Tools}(BWAT) berbasis \textit{Android}. Data kajian yang digunakan dalam penelitian ini terfokus pada proses pencatatan pasien berobat penanganan luka kronis. Proses pengembangan sistem ini menggunakan metode \textit{Scrum} dan seluruh aplikasi yang dibuat menggunakan bahasa pemrograman \textit{Kotlin} dan diintegrasi dengan penelitian-penelitian sebelumnya. Hasil akhir dari penelitian ini berupa aplikasi pengkajian luka yang membantu perawat dalam melakukan pencatatan proses penyembuhan luka pasien dan pasien dapat melihat progres penyembuhan dirinya sendiri.
\newline
\newline
\noindent \textbf{Kata kunci}: \textit{android, aplikasi, luka kronis, BWAT, scrum, pencatatan, integrasi}
\end{spacing}